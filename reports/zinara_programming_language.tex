\documentclass[12pt, spanish]{report}
\usepackage[spanish, activeacute]{babel}
\usepackage[latin1]{inputenc}
\renewcommand{\baselinestretch}{1.3}
\usepackage{graphicx}

\begin{document}
\title{Universidad Sim�n Bol�var \\ Lenguajes de Programaci�n II \\ Zinara}
\author{
  Daniel Barreto - \#04-36723 \texttt{<daniel.barreto.n@gmail.com>} \\
  Germ�n Jaber - \#06-39743 \texttt{<germy-boy@gmail.com>}
}
\maketitle

\tableofcontents

\newpage

\chapter{Introducci�n}
\label{chap:intr}



\chapter{Dise�o del Lenguaje}
\label{chap:diseno}


\section{Filosof�a y paradigma de programaci�n}
\label{sec:filpar}


\section{Dise�o de Tipos}

\label{sec:tipos}

\subsection{Tipos b�sicos}
\label{sec:tiposbasicos}


\section{Sintaxis y Sem�ntica}
\label{sec:synsem}

\subsection{Declaraciones}
\label{sec:decl}

\subsection{Instrucciones}
\label{sec:instr}

\subsection{Expresiones}
\label{sec:expr}

\subsection{Funciones y Procedimientos}
\label{sec:func}



\chapter{Implementaci�n}
\label{chap:impl}


\section{Herramientas utilizadas}
\label{sec:herramientas}


\section{Requerimientos para compilar}
\label{sec:requ}


\section{El c�digo}
\label{sec:codigo}

\subsection{Paquetes}
\label{sec:paquetes}

\subsection{�rbol sint�ctico abstracto}
\label{sec:ast}

\subsection{Tabla de s�mbolos}
\label{sec:st}



\chapter{Estado actual del lenguaje}
\label{chap:estado}



\end{document}