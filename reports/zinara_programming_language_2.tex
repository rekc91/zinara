\documentclass[12pt, spanish]{report}
\usepackage[latin1]{inputenc}
\usepackage[spanish]{babel}
\renewcommand{\baselinestretch}{1.3}
\usepackage{graphicx}

\begin{document}
\title{Universidad Simón Bolívar \\ Lenguajes de Programación II \\ Zinara}
\author{
  Daniel Barreto - \#04-36723 \texttt{<daniel.barreto.n@gmail.com>} \\
  Germán Jaber - \#06-39743 \texttt{<german@ac.labf.usb.ve>}
}
\maketitle

\tableofcontents

\newpage


\chapter{Diseño del Lenguaje}
\label{chap:diseno}

Este capítulo describe cuales han sido las ideas principales que han
sido tomadas en cuenta para realizar el diseño general de
\emph{\textbf{Zinara}} y posteriormente se detalla como han sido
tomadas estas ideas para ser plasmadas en la sintaxis y semántica del
lenguaje.


\section{Filosofía y paradigma de programación}
\label{sec:filpar}

\textbf{\emph{Zinara}} esta orientado a ser un lenguaje simple y
sencillo de leer y aprender, cuyo paradigma de programación será
únicamente imperativo.\\

Los principales lineamientos tomados a la hora de desarrollar el
diseño del lenguaje han sido:

\begin{itemize}
\item \emph{Lo explicito es mejor que lo implicito:}\\
  \emph{\textbf{Zinara}} es un lenguaje fuertemente y estáticamente
  tipeado, lo que le permite al programador establecer explícitamente
  todos los tipos de las variables de un programa, y asi poder
  realizar a tiempo de compilación gran parte del chequeo del
  funcionamiento del mismo.

\item \emph{La legibilidad es importante:}\\
  La sintaxis del lenguaje intentará ser lo mas simple y minimalista
  posible, sin perder poder de expresividad.

\item \emph{\textbf{Zinara} debe ayudar al programador a sentirse
    cómodo:}\\
  El diseño de las estructuras y herramientas que ofrece
  \emph{\textbf{Zinara}} estará orientado a brindar flexibilidad y
  poder de expresión al usuario, sacrificando para esto parte del
  rendimiento de los programas escritos en el lenguaje.
\end{itemize}


\section{Diseño de Tipos}
\label{sec:tipos}

Para esta entrega sólo han sido desarrollados los tipos básicos de
\emph{\textbf{Zinara}}, los cuales son descritos a continuación.

\subsection{Tipos básicos}
\label{sec:tiposbasicos}

Los tipos b\'asicos del lenguaje son:

\begin{itemize}
\item \textbf{Int}: Entero de 4 \emph{bytes}.
\item \textbf{Float}: Flotante de 4 \emph{bytes}.
\item \textbf{Bool}: Booleano de 1 \emph{byte}.
\item \textbf{Char}: Caracter de 1 \emph{byte}.
\end{itemize}

A partir de estos tipos básicos se construirán los tipos compuestos
que se mostrarán en las siguientes entregas.

\section{Sintaxis y Sem\'antica}
\label{sec:synsem}

Como se mencionó anteriormente, \emph{\textbf{Zinara}} trata de ser un
lenguaje que sea facilmente leíble, por lo tanto su sintaxis está
pensada para ser simple y minimalista, buscando tener la menor
cantidad de adornos posibles, a diferencia de los lenguajes de
programación tradicionales\footnote{C/C++, Java, Pascal, entre otros}
que cuentan con una gran cantidad de signos de puntuación como
delimitadores de instrucciones o declaraciones.\\

Para conseguir este objetivo se tomaron las siguientes decisiones en
cuanto a la sintaxis del lenguaje:

\begin{itemize}
\item Las instrucciones de un programa estarán delimitadas por un
  salto de línea, sin algún tipo de signo de puntuación divisorio.

\item Los bloques de código (que serán detallados más adelante)
  estarán delimitados únicamente por una palabra clave que represente
  dicho bloque, condiciones extras si son requeridas, y la palabra
  clave de finalización de bloque de código: ``end''.

\item Cuando sea posible y tenga sentido hacerlo, se dará preferencia
  a usar palabras del idioma inglés como palabras clave en donde los
  lenguajes de programación tradicionales utilizan signos de
  puntuación.
\end{itemize}

A continuación se describe como estan diseñadas las partes principales
de la sintaxis del lenguaje. Para facilitar la explicación se
establecen los siguientes lineamientos:

\begin{itemize}
\item Una secuencia de car\'acteres v\'alida es aquella que pertenece al
      lenguaje regular texttt{[a-zA-Z_0-9:()\n ]+}.

\item Un elemento es una secuencia de car\'acteres v\'alida encerrada
      entre corchetes, entre \texttt{<} y \texttt{>}, entre numerales no encerrada.

\item Toda secuencia de car\'acteres v\'alida se tomar\'a como literal a menos
      que est\'e encerrada entre \texttt{<} y \texttt{>} o entre corchetes.

\item Uno o m\'as elementos pueden estar encerrados entre corchetes,
      esto querr\'a decir que ese grupo de elementos, en ese order, es
      opcional.

\item Si uno o m\'as elementos est\'an encerrados entre numerales,
      quiere decir que esos elementos deben ser considerados un solo 
      elemento.

\item Si dos elementos est\'an separados por \texttt{|} indica que uno y
      solo uno de esos elementos puede aparecer en ese punto.

\item \texttt{<tipo>} representa un tipo cualquiera del lenguaje, entre ellos
  los tipos básicos explicados anteriormente en la sección
  \ref{sec:tiposbasicos}.

\item \texttt{<identificador>}, \texttt{<id1>}, \texttt{<id2>}...
  \texttt{<idN>} representan identificadores del lenguaje.

\item \texttt{<expresion>}, \texttt{<expr1>}, \texttt{<expr2>}...
  \texttt{exprN} representan expresiones cualquiera del lenguaje, como
  son explicadas en la sección \ref{sec:expr}.

\item \texttt{<code_block>}, \texttt{<code_block1>}, \texttt{<code_block2>}...
  \texttt{code_blockN} representan bloques de c\'odigo del lenguaje,
  estos bloques de c\'odigo se contruyen secuenciado intrucciones y
      estructuras de control del lenguaje. La secuenciaci\'on esta 
  son explicadas en la sección \ref{sec:expr}.

\item \texttt{<expresion>}, \texttt{<expr1>}, \texttt{<expr2>}...
  \texttt{exprN} representan expresiones cualquiera del lenguaje, como
  son explicadas en la sección \ref{sec:expr}.

\item La cantidad de espacios que deben separar a dos palabras clave es
      arbitraria, pero siempre mayor a cero.
\end{itemize}

\subsection{Identificadores}
\label{sec:ident}
Los identificadores son aquellas frases que pertenecen al lenguaje
regular \texttt{[a-zA-Z]][a-zA-Z0-9_']}. Es decir, los identificadores
deben comenzar por una letra, a partir de ese punto cualquier
combinaci\'on de letras may\'usculas y min\'usculas, n\'umeros,
\texttt{_} y \texttt{'} (comillas simples) es v\'alida. 

\subsection{Declaraciones}
\label{sec:decl}

Las declaraciones se pueden realizar de dos formas:\\

\textbf{Una a la vez}:
\begin{verbatim}
[var] <tipo> <identificador> [= <expresion>]
\end{verbatim}

\textbf{o de forma múltiple}:
\begin{verbatim}
[var] <tipo> <id1>, <id2>, ..., <idN> [= <expr1>, <expr2>, ..., <exprN>]
\end{verbatim}

La palabra clave \texttt{var} denota si el identificador, o los
indentificadores, que estan siendo declarados se comportarán como
variables en el lenguaje de programación. Si \texttt{var} se omite, el
identificador (o identificadores) será declarado como constante, y una
vez que le sea asignado un valor de una expresión, éste será
inmutable. Las palabras clave para los tipos son:
\begin{itemize}
\item Int
\item Float
\item Bool
\item Char
\end{itemize}

% PORQUE?!

Existe un tercer tipo de declaración: \emph{Declaración de funciones},
el cual es explicado en la sección \ref{sec:func}.

\subsection{Instrucciones}
\label{sec:instr}

Las instrucciones del lenguaje que se pueden realizar sobre los tipos
básicos definidos en la sección \ref{sec:tiposbasicos} son las siguientes:

\begin{itemize}
\item Secuenciación
\item Asignación
\item Return
\item Estructuras de control (explicadas con detalle en la sección \ref{sec:ecntrl})
\item Impresi\'on
\end{itemize}

Adicionalmente, las declaraciones de identificadores o funciones
también cuentan como instrucciones del lenguaje y pueden ir en cualquier
lugar donde pueda ir una funci\'on.

\subsubsection{Secuenciaci\'on}
Se secuenci\'an varias instrucciones y estructuras de control separandolas con saltos
de l\'inea. No importa que la \'ultima instrucci\'on tenga
tambi\'en un salto de l\'inea.

\subsubsection{Asignaci\'on}
Sintaxis:
Las asignaciones se pueden realizar de dos formas:\\

\textbf{Una a la vez}:
\begin{verbatim}
<identificador> [= <expresion>]
\end{verbatim}

\textbf{O de forma múltiple}:
\begin{verbatim}
<id1>, <id2>, ..., <idN> [= <expr1>, <expr2>, ..., <exprN>]
\end{verbatim}

\subsubsection{Return}
\label{sec:return}
Sintaxis:
\begin{verbatim}
return <expr>
\end{verbatim}

Esta instrucci\'on solo puede ser usada dentro de una funci\'on. Lo
que hace es detener el flujo de la funci\'on y hacer que esta devuelva
la expresión \texttt{<expr>}.

\subsubsection{Break}
\label{sec:return}
Sintaxis:
\begin{verbatim}
break
\end{verbatim}

Esta instrucci\'on solo puede ser usada dentro de una
iteraci\'on. Ocasiona que la iteraci\'on deje de ejecutarse.
Si la estructura est\'a anidada dentro de otra iteraci\'on, 
solo la estructura de control m\'as interna termina.

\subsubsection{Continue}
\label{sec:return}
Sintaxis:
\begin{verbatim}
continue
\end{verbatim}

Esta instrucci\'on solo puede ser usada dentro de una
iteraci\'on. Ocasiona que las instrucciones pertenecientes a la
iteraci\'on pero posteriores a la instrucci\'on sean ignoradas y se
comienza un nuevo ciclo en la estructura de control iterativa.
Al igual que la instrucci\'on \texttt{break}, si la iteraci\'on est\'a
anidada dentro de otra iteraci\'on, solo la estructura m\'as interna se
ve afectada por la instrucci\'on.

\subsubsection{Impresi\'on}
Sintaxis:
\begin{verbatim}
print ( <expr> )
\end{verbatim}

Imprime la expresion entre par\'entesis a la salida est\'andar. \texttt{expr}
puede tener cualquier tipo del lenguaje, pero no un tipo definido por el usuario.

\subsection{Estructuras de control}
\label{sec:ecntrl}
\emph{\textbf{Zinara}} tiene 4 estructuras de control
\begin{itemize}
\item If: para ramificar el flujo del programa
\item While: para ciclos gen\'ericos
\item Cycle: tambi\'en para ciclos gen\'ericos
\item For: para iterar sobre listas y tuplas
\end{itemize}

Los bloques de las estructuras de control est\'an pueden ser finalizados
con `end` o con un terminador de bloque de control particular al tipo de
estructura de control que se este construyendo. Por ejemplo, un if puede
finalizarse con `end` o `endif`, un for con `end` o `endfor` y asi
sucesivamente.
Estas dos opciones le permiten al programador elegir entre darle m\'as
legibilidad al c\'odigo, o simplemente tipear menos car\'acteres para
finalizar sus bloques de control de flujo.
Ahora se revisar\'a la sintaxis de las estructuras de control, pero
primero se dar\'a una breve explicaci\'on de la notacion usada para
describir la sintaxis.
\begin{itemize}\
\end{itemize}

\subsubsection{If}
Sintaxis:
\begin{verbatim}
   if <expr> :
       <code_block1>
  [elif <expr> :
       <code_block2>]
   .
   .
   .
  [else :
       <code_blockN>]
   end|endif
\end{verbatim}

\subsubsection{While}
Sintaxis:
\begin{verbatim}
   while <expr> :
       <code_block>
   end|endwhile
\end{verbatim}

\subsubsection{Cycle}
Sintaxis:
\begin{verbatim}
   cycle:
     [code_block]
     [on <expr1> :
          <code_block1>]
     [on <expr2> :
          <code_block2>]
      .
      .
      .
      # default :
         <code_blockN> #|
      # on <exprn> : 
         <code_blockN> #
   end|endcycle
\end{verbatim}
Esta estructura de control iterativa recorrera cada una de las clausulas
\texttt{on} probando el valor de la expresi\'on, tomar\'a la primera que sea cierta
y cuando termine la ejecuci\'on del bloque de codigo asociado a esa
clausula volver\'a a recorrerlas. La estructura para cuando ninguna
clausula se cumple. Opcionalmente se puede agregar un bloque de c\'odigo
al pricipio de todas las clausulas \texttt{on} que se ejecutar\'a al
pricipio de todos los ciclos.


\subsubsection{For}
Sintaxis:
\begin{verbatim}
   for <id> in <expr> :
       <code_block>
   end
\end{verbatim}

Esta estructura esta dise\~nada para recorrer listas\footnote{Las
  listas son tipos de datos compuestos que serán explicadas en una
  siguiente entrega} del lenguaje, por lo que el valor de
\texttt{expr} s\'olo puede ser una lista. El elemento actual de la
iteraci\'on solo se puede accesar a trav\'es del identificador
\texttt{id} dentro del ciclo. Ni el elemento iterado \texttt{expr} ni
el \'indice \texttt{id} pueden ser modificados dentro de esta estructura
de control.\\


\subsection{Operadores}
\label{sec:ops}

Los siguientes son operadores definidos en \emph{\textbf{Zinara}} para
interactuar con los tipos básicos definidos en la sección
\ref{sec:tiposbasicos}.

\subsubsection{Operadores Aritm\'eticos}
\begin{itemize}
\item Suma (\texttt{+})
\item Resta    (\texttt{-})
\item Producto (\texttt{*})
\item Divisi\'on  (\texttt{/})
\item M\'odulo (\texttt{\%})
\item Potencia (\texttt{\^})
\item Menos unario (\texttt{-})
\end{itemize}

\subsubsection{Operadores Relacionales}
\begin{itemize}
\item Mayor         (\texttt{>})
\item Menor         (\texttt{<})
\item Mayor o igual (\texttt{>=})
\item Menor o igual (\texttt{<=})
\item Igual (superficial) (\texttt{==})
\item Igual (profunda)    (\texttt{<=>})
\item Negaci\'on      (\texttt{!})
\end{itemize}

\subsubsection{Operadores L\'ogicos}
\begin{itemize}
\item And con corto-circuito (\texttt{AND})
\item And sin corto-circuito (\texttt{\&})
\item Or con corto-circuito  (\texttt{OR})
\item Or sin corto-circuito  (\texttt{|})
\item Xor (\texttt{XOR})
\end{itemize}

Todos los operadores asocian a la izquierda. Sus precedencias son:
\begin{itemize}
\item \texttt{AND}, \texttt{\&}, \texttt{OR}, \texttt{|}, \texttt{XOR}
\item \texttt{<} , \texttt{>} , \texttt{<=} , \texttt{>=} , \texttt{==} , \texttt{<=>} , \texttt{!}
\item \texttt{\-} (Unario)
\item \texttt{\^}
\item \texttt{*} ,  \texttt{/}, \texttt{\%}
\item \texttt{+}, \texttt{-} (Binario)
\end{itemize}

\subsubsection{Compatiblidad de tipos}
Se decidi\'o no usar un n\'umero elevado de coerciones para hacer el
sistema de tipos m\'as \'util al momento de prevenir el uso incorrecto
de variables y para ayudar al programador escribir una abstracci\'on del
problema m\'as precisa.
A continuaci\'on presentamos las tablas de compatibilidad de
tipos. Estas muestran el resultado de usar un operador binario sobre
todas las posibles combinaciones de tipos b\'asicos del lenguaje 

\begin{itemize}
\item
\begin{tabular}{|c||cccc|}
	\hline
+      &  Int  & Float & Char  & Bool  \\
	\hline \hline
Int    & Int   & Float & \emph{Error} & \emph{Error} \\
Float  & Float & Float & \emph{Error} & \emph{Error} \\
Char   & \emph{Error} & \emph{Error} & \emph{Error} & \emph{Error} \\
Bool   & \emph{Error} & \emph{Error} & \emph{Error} & \emph{Error} \\
	\hline
\end{tabular}

\item
\begin{tabular}{|c||cccc|}
	\hline
-      &  Int  & Float & Char  & Bool  \\
	\hline \hline
Int    & Int   & Float & \emph{Error} & \emph{Error} \\
Float  & Float & Float & \emph{Error} & \emph{Error} \\
Char   & \emph{Error} & \emph{Error} & \emph{Error} & \emph{Error} \\
Bool   & \emph{Error} & \emph{Error} & \emph{Error} & \emph{Error} \\
	\hline
\end{tabular}

\item
\begin{tabular}{|c||cccc|}
	\hline
*      &  Int  & Float & Char  & Bool  \\
	\hline \hline
Int    &  Int         & Float & \emph{Error} & \emph{Error} \\
Float  & Float        & Float & \emph{Error} & \emph{Error} \\
Char   & \emph{Error} & \emph{Error} & \emph{Error} & \emph{Error} \\
Bool   & \emph{Error} & \emph{Error} & \emph{Error} & \emph{Error} \\
	\hline
\end{tabular}

\item
\begin{tabular}{|c||cccc|}
	\hline
/      &  Int  & Float & Char  & Bool  \\
	\hline \hline
Int    &  Int  & Float & \emph{Error} & \emph{Error} \\
Float  & Float & Float & \emph{Error} & \emph{Error} \\
Char   & \emph{Error} & \emph{Error} & \emph{Error} & \emph{Error} \\
Bool   & \emph{Error} & \emph{Error} & \emph{Error} & \emph{Error} \\
	\hline
\end{tabular}

\item
\begin{tabular}{|c||cccc|}
	\hline
\%     &  Int  & Float & Char  & Bool  \\
	\hline \hline
Int    &  Int  & Float & \emph{Error} & \emph{Error} \\
Float  & Float & Float & \emph{Error} & \emph{Error} \\
Char   & \emph{Error} & \emph{Error} & \emph{Error} & \emph{Error} \\
Bool   & \emph{Error} & \emph{Error} & \emph{Error} & \emph{Error} \\
	\hline
\end{tabular}

\item
\begin{tabular}{|c||cccc|}
	\hline
\texttt{\^}\     &  Int  & Float & Char  & Bool  \\
	\hline \hline
Int    &  Int  & Float & \emph{Error} & \emph{Error} \\
Float  & Float & Float & \emph{Error} & \emph{Error} \\
Char   & \emph{Error} & \emph{Error} & \emph{Error} & \emph{Error} \\
Bool   & \emph{Error} & \emph{Error} & \emph{Error} & \emph{Error} \\
	\hline
\end{tabular}
Para flotantes y enteros esta operaci\'on representa el m\'odulo de la
divisi\'on de los valores.

\item
\begin{tabular}{|c||cccc|}
	\hline
\texttt{<,>,<=,>=}&  Int  & Float & Char  & Bool  \\
	\hline \hline
Int      &  Bool &  Bool & \emph{Error} & \emph{Error} \\
Float    &  Bool &  Bool & \emph{Error} & \emph{Error} \\
Char     & \emph{Error} & \emph{Error} & \emph{Error} & \emph{Error} \\
Bool     & \emph{Error} & \emph{Error} & \emph{Error} & \emph{Error} \\
	\hline
\end{tabular}

\item
\begin{tabular}{|c||cccc|}
	    \hline
AND,OR,\texttt{|},\& &  Int  & Float & Char  & Bool  \\ 
	    \hline \hline
Int         & \emph{Error} & \emph{Error} & \emph{Error} & \emph{Error} \\
Float       & \emph{Error} & \emph{Error} & \emph{Error} & \emph{Error} \\
Char        & \emph{Error} & \emph{Error} & \emph{Error} & \emph{Error} \\
Bool        & \emph{Error} & \emph{Error} & \emph{Error} & Bool \\
	\hline
\end{tabular}

\item
\begin{tabular}{|c||cccc|}
 \hline
!       &  Int  & Float & Char  & Bool  \\ 
 \hline \hline
        & \emph{Error} & \emph{Error} & \emph{Error} & \emph{Bool} \\
	\hline
\end{tabular}
\end{itemize}

\subsection{Expresiones}
\label{sec:expr}

Las expresiones tienen un valor y un tipo asociado. tanto el valor
como el tipo de una expresión pueden ser determinados en función de
los elementos que formen parte de la misma.

Para esta entrega sólo se especificarán las expresiones que se pueden
realizar con los tipos básicos de \emph{\textbf{Zinara}}. Se detallan
a continuación:

\begin{itemize}
\item \textbf{Expresión Binaria:} Es una expresion que está
      determinada por dos expresiones cualquiera y un operador que las
      relaciona. El operador puede ser cualquiera de los operadores
      binarios detallados en la sección \ref{sec:ops}.
      
 \item \textbf{Expresión Unaria:} Es una expresión que se encuentra
       formada por algúna otra expresión cualquiera y un operador que actúa
       sobre ella. Análogamente que con las expresiones binarias, el
       operador puede ser cualquiera de los operadores unarios detallados
       en la sección \ref{sec:ops}.
       
 \item Cualquier literal de los tipos básicos del lenguaje es una
       expresión. Por ejemplo \texttt{42} en el caso de los enteros,
       \texttt{3.5} en el caso de los flotantes, \texttt{'a'} en el caso de
       los caracteres y \texttt{True} en el caso de los booleanos.
       
 \item Cualquier identificador ya declarado de un programa puede ser
       usado como una expresión que tiene el mismo tipo de la declaración
       del identificador y su valor es el valor que almacene dicho
       identificador. El comportamiento del lenguaje si se usa un
       identificador al cual no se le ha asignado ning\'un valor esta,
       por los momentos, indefinido.
       
 \item Cualquier llamada a una función ya declarada es una expresión
       que tiene como valor resultante el valor devuelto por la función, y
       su tipo es el tipo de retorno de la función.
       %CHEQUEARRRRRRRRRRRRRRRRRRRRRRRRRRRRRRRRRRR
\end{itemize}

\subsection{Funciones}
\label{sec:func}

Las funciones en \emph{\textbf{Zinara}} son declaradas de la siguiente
forma:

\begin{verbatim}
def <identificador>(<lista_de_argumentos>) returns <<tipoR>>:
[codigo de la funcion]
end[def]
\end{verbatim}

Dónde \texttt{<tipoR>} es el tipo de retorno, y
\texttt{<lista\_de\_argumentos>} es la lista de los argumentos de la
función definida de la siguiente forma:

\begin{verbatim}
[copy] <arg1> as <tipo1>, ..., [copy] <argN> as <tipoN>
\end{verbatim}

Las funciones recibirán todos sus parámetros por referencia. En la
declaración de una función se tienen que declarar explicitamente los
tipos de cada uno de los argumentos y adicionalmente se proporciona la
inclusión opcional de la palabra clave \texttt{copy} antes de la
declaración de algún argumento para forzar al lenguaje a hacer una
copia del valor de dicho argumento y no modificar su referencia
durante la ejecución del código de la función.\\

Las funciónes tambien pueden ser usadas como expresiones, llamándolas
de la siguiente forma:

\begin{verbatim}
<identificador>(<expr1>, ..., <exprN>)
\end{verbatim}

\chapter{Implementación}
\label{chap:impl}

Este capítulo describe brevemente las herramientas que fueron
utilizadas para la realización del analizador sintáctico
(\emph{parser}) y el analizador lexicográfico (\emph{lexer}) de
\emph{\textbf{Zinara}}, así como el resto de las herramientas que han
sido utilizadas en todo el desarrollo del lenguaje.

\section{Herramientas utilizadas}
\label{sec:herramientas}

\begin{itemize}
\item Para generar el analizador lexicográfico se utilizó
  \emph{JFlex}\cite{jflex}, una herramienta que recibe una
  especificación de \emph{tokens} o palabras claves y genera código en
  \emph{Java} que se encarga de hacer el reconocimiento de dichos
  \emph{tokens} dada una entrada de texto cualquiera.

\item Para el analizador sintáctico se utilizó la herramienta
  \emph{JavaCUP}\cite{javacup}, un generador de \emph{parsers} LALR(1)
  que recibe una especificación de gramática con sintaxis similar al
  conocido \emph{YACC} y genera el autómata encargado en reconocer
  dicha gramática en código \emph{Java}.\\

  Ambas herramientas están bien documentadas y ofrecen facilidades
  para interactuar y comunicarse entre ellas, lo cual tuvo un gran
  peso en la decisión de elegirlas para el desarrollo de
  \emph{\textbf{Zinara}}.

\item Para simplificar el proceso de compilación se utiliza
  \emph{Apache Ant}\cite{ant}, una herramienta de compilación para
  compilación dirigida en \emph{Java}.

\item Para el manejo colaborativo del código se utiliza
  \emph{GIT}\cite{git}, un controlador de versiones distribuido
  diseñado para manejar grandes y pequeños proyectos de forma rápida.
\end{itemize}

\section{Requerimientos}
\label{sec:requ}

Para la compilación y ejecución del proyecto es necesario cumplir con
los siguientes requerimientos mínimos:

\begin{itemize}
\item Máquina virtual y kit de desarrollo de \emph{Java 1.4.2}.
\item Instalación estándar de \emph{JFlex}.
\item Instalación estándar de \emph{JavaCup}.
\item Instalación estándar de \emph{Apache Ant} (únicamente para
  compilación).
\end{itemize}

\section{Detalles de implementación}
\label{sec:codigo}

Para la realización del proyecto se organizó el código fuente en un
paquete de java llamado \texttt{zinara} el cual esta estructurado de
la siguiente manera:

\begin{itemize}
\item \texttt{zinara.ast}: Paquete dónde se encuentran las clases
  necesarias para representar el árbol sintáctico abstracto y la tabla
  de símbolos. Estas clases son explicadas con más detalle a
  continuación en las secciones \ref{sec:ast} y \ref{sec:st}.

\item \texttt{zinara.lexer}: Paquete dónde se encuentra el archivo
  \texttt{lexer.flex}, que representa la descripción lexicográfica de
  los \emph{tokens} de \emph{\textbf{Zinara}}. En este paquete se
  guarda un archivo llamado \texttt{Scanner.java} que contiene el
  código generado por la herramienta \emph{JFlex} al ser aplicada
  sobre el archivo \texttt{lexer.flex}.

\item \texttt{zinara.parser}: Paquete que contiene el archivo
  \texttt{parser.cup} que describe la gramática de
  \emph{\textbf{Zinara}}. Análogamente al paquete
  \texttt{zinara.lexer}, en \texttt{zinara.parser} se guardan los
  archivos \texttt{parser.java} y \texttt{sym.java} que son generados
  por la aplicación de la herramienta \emph{JavaCUP} sobre el archivo
  \texttt{parser.cup}.
\end{itemize}

La clase principal de ejecución del proyecto se encuentra en
\texttt{zinara.Main}, la cual espera recibir como parámetro de entrada
un archivo de texto que contenga código escrito para
\emph{\textbf{Zinara}}. Esta clase se encarga de invocar tanto al
analizador lexicográfico como al analizador sintáctico y ejecutarlos
adecuadamente sobre el contenido del archivo recibido. Actualmente
ésta clase principal no realiza ningún tipo de chequeo extra a parte
de los mencionados.

\subsection{Árbol sintáctico abstracto}
\label{sec:ast}

El árbol sintáctico abstracto se encuentra implementado como una
jerarquía de clases que establecen la representación sintáctica de
cada uno de los elementos del lenguaje.\\

Las clases del árbol sintáctico abstracto están dividas en tres
conjuntos principales: \emph{expressions} para las expresiones,
\emph{instructions} para las instrucciones y \emph{types} para los
tipos. Dentro de cada una de las dos primeras divisiones se encuentran
clases que representan sintácticamente las estructuras explicadas en
las secciones \ref{sec:expr} y \ref{sec:instr} respectivamente.\\

En el conjunto \emph{types} se encuentran clases que representan los 4
tipos básicos mencionados en la sección \ref{sec:tiposbasicos}, y una
clase para representar el tipo de una función que es compuesto entre
los tipos de sus argumentos y su tipo de retorno.

\subsection{Tabla de s\'imbolos}
\label{sec:st}

Las clases que manejan la tabla de s\'imbolos estan en los archivos
\texttt{Symtable.java} y \texttt{SymValue.java}, y se encuentran en el paquete
\texttt{zinara.ast} 

\begin{itemize}
\item \texttt{Symtable.java}: es un wrapper para HashTable. Simplemente
  da nombres semanticamente mas coherentes con el manejo de s\'imbolos
  que los de HashTable. Los pares key/value guardados en la tabla de
  hash con del tipo String/SymValue.

\item \texttt{SymValue.java}: es un contenedor que guarda un Object,
  un Type y un boolean. El Object es el valor del s\'imbolo, el Type
  simplemente indica el tipo del objeto y el boolean indica si el
  s\'imbolo es variable o no.
\end{itemize}

\section{Código adjunto del proyecto}
\label{sec:correr}

Este proyecto tiene adjunto un archivo comprimido llamado\\

\texttt{zinara-barreto-jaber-entrega1.tar.gz}\\

y dentro de éste archivo se encuentra una única carpeta llamada
\textbf{Zinara-Barreto-Jaber/}. Una vez descomprimido el archivo se
encontrará la siguiente estructura de directorios:

\begin{itemize}
\item \textbf{labs/}: Directorio donde se guardan los laboratorios
  asignados en clases.
\item \textbf{project/}: Directorio principal del proyecto, donde se
  encuentra el codigo fuente y los binarios para la ejecución.
  \begin{itemize}
  \item \textbf{project/bin/}: Directorio donde se guardan los
    archivos ejecutables para correr el proyecto.
  \item \textbf{project/classes/}: Directorio donde en encuentran los
    archivos \texttt{.class} del proyecto.
  \item \textbf{project/lib/}: Directorio donde se guardan las
    librerias externas usadas en el desarrollo del proyecto. Para esta
    entrega no se a agregado ninguna.
  \item \textbf{project/src/}: Directorio donde se encuentra el código
    fuente del proyecto.
  \end{itemize}
\item \textbf{reports/}: Directorio donde se guarda el código fuente
  del presente informe, así como su versión en \emph{pdf}.
\end{itemize}

\subsection{Ejecución}
\label{sec:zejecucion}
Los pasos para ejecutar el proyecto en una consola UNIX son los
siguientes:

\begin{verbatim}
$> cd Zinara-Barreto-Jaber/project/bin/
$> java -jar zinara.jar archivo_de_prueba.zn
\end{verbatim}

Opcionalmente se puede ejecutar el proyecto desde el directorio
\texttt{project/classes/}, realizando los siguientes pasos:

\begin{verbatim}
$> cd Zinara-Barreto-Jaber/project/classes/
$> java zinara.Main archivo_de_prueba.zn
\end{verbatim}

Es importante recordar que para la ejecución del proyecto, dada la
elección de cualquiera de los dos métodos de ejecución explicados
anteriormente, es necesario tener bien establecidas las variables de
ambiente requeridas por las herramientas utilizadas. Ir a
\cite{javacup} y \cite{jflex} para más información.

\subsection{Compilación}
\label{sec:zcompilacion}
Para compilar \emph{\textbf{Zinara}}, una vez que haya sido editado
algún archivo del código fuente, se ejecutan las siguientes líneas de
comando en una consola UNIX.

\begin{verbatim}
$> cd Zinara-Barreto-Jaber/project/
$> ant dist
\end{verbatim}

Esto se encargará de compilar y empaquetar adecuadamente el proyecto,
y un ejecutable \texttt{.jar} será colocado dentro de la carpeta
\texttt{bin/}.\\

Es importante destacar que si se editan alguno de los archivos
\texttt{parser.cup} o \texttt{lexer.flex}, es necesario ejecutar sobre
ellos la herramienta \emph{JavaCUP} o la herramienta \emph{JFlex}
según sea el caso, para poder generar los archivos \texttt{.java}
correspondientes.\\

La ejecución de las herramientas \emph{JavaCUP} y \emph{JFlex} no será
documentada en este informe pués existen formas diversas de hacerlo, y
varían según el sistema operativo donde se ejecute y la instalación
que se haya realizado de ellas en cada computadora. Para más
información referirse a \cite{javacup} y \cite{jflex}.\\

En \cite{github} se encuentra un repositorio con el mismo código
fuente adjunto (posiblemente más actualizado) y el código fuente de
las herramientas que estan siendo utilizadas para el desarrollo de
\emph{\textbf{Zinara}}.

\chapter{Estado actual del lenguaje}
\label{chap:estado}

Actualmente \emph{\textbf{Zinara}} cuenta con un funcionamiento
correcto en su analizador lexicográfico y su analizador sintáctico. A
parte de esto también construye el árbol sintáctico abstracto y la
tabla de símbolos de un programa dado. Las límitaciones de ésto último
se encuentran en el manejo de la tabla de símbolos para las funciones,
el cual todavia no se encuentra completamente implementado. Además,
la tabla de símbolos no se comporta bien para los \texttt{for} del
lenguaje, dado que no toma en cuenta la variable de iteración.\\

El chequeo de tipos se encuentra inestable y hemos decidido no
incluirlo en el código adjunto, sin embargo existen ya adelantos que
pueden ser observados en \cite{github}.

\bibliographystyle{plain}
\bibliography{bibliography}

\end{document}