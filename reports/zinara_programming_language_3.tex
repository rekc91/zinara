\documentclass[12pt, spanish]{report}
%\usepackage[latin1]{inputenc}
\usepackage[spanish]{babel}
\renewcommand{\baselinestretch}{1.3}
\usepackage{graphicx}

\begin{document}
\title{Universidad Sim\'on Bol\'ivar \\ Lenguajes de Programaci\'on III \\ Zinara}
\author{
  Daniel Barreto - \#04-36723 \texttt{<daniel.barreto.n@gmail.com>} \\
  Germ\'an Jaber - \#06-39743 \texttt{<german@ac.labf.usb.ve>}
}
\maketitle

\tableofcontents

\newpage
\chapter{Filosof\'ia y paradigma de programaci\'on}
\label{chap:diseno}

Este cap\'itulo describe cuales han sido las ideas principales que han
sido tomadas en cuenta para realizar el dise\~no general de
\emph{\textbf{Zinara}} y posteriormente se detalla como han sido
tomadas estas ideas para ser plasmadas en la sintaxis y sem\'antica del
lenguaje.

\textbf{\emph{Zinara}} esta orientado a ser un lenguaje simple y
sencillo de leer y aprender, cuyo paradigma de programaci\'on ser\'a
\'unicamente imperativo.\\

Los principales lineamientos tomados a la hora de desarrollar el
dise\~no del lenguaje han sido:

\begin{itemize}
\item \emph{Lo expl\'icito es mejor que lo impl\'icito:}\\
  \emph{\textbf{Zinara}} es un lenguaje fuertemente y est\'aticamente
  tipeado, lo que le permite al programador establecer expl\'icitamente
  todos los tipos de las variables de un programa, y asi poder
  realizar a tiempo de compilaci\'on gran parte del chequeo del
  funcionamiento del mismo.

\item \emph{La legibilidad es importante:}\\
  La sintaxis del lenguaje intentar\'a ser lo mas simple y minimalista
  posible, sin perder poder de expresividad.

\item \emph{\textbf{Zinara} debe ayudar al programador a sentirse
    c\'omodo:}\\
  El dise\~no de las estructuras y herramientas que ofrece
  \emph{\textbf{Zinara}} estar\'a orientado a brindar flexibilidad y
  poder de expresi\'on al usuario, sacrificando para esto parte del
  rendimiento de los programas escritos en el lenguaje.
\end{itemize}

Como se menciona anteriormente, \emph{\textbf{Zinara}} trata de ser un
lenguaje que sea facilmente le\'ible, por lo tanto su sintaxis est\'a
pensada para ser simple y minimalista, buscando tener la menor
cantidad de adornos posibles, a diferencia de los lenguajes de
programaci\'on tradicionales\footnote{C/C++, Java, Pascal, entre
  otros} que cuentan con una gran cantidad de signos de puntuaci\'on
como delimitadores de instrucciones o declaraciones.\\

Para conseguir este objetivo se tomaron las siguientes decisiones en
cuanto a la sintaxis del lenguaje:

\begin{itemize}
\item Las instrucciones de un programa estar\'an delimitadas por un
  salto de l\'inea, sin ning\'un tipo de signo de puntuaci\'on
  divisorio.

\item Las estructuras de control (que ser\'an detalladas m\'as
  adelante) estar\'an delimitadas \'unicamente por una palabra clave
  que representa dicha estructura, condiciones extras si son
  requeridas, y la palabra clave de finalizaci\'on de estructura de
  control: ``end''.

\item Cuando sea posible y tenga sentido hacerlo, se dar\'a
  preferencia a usar palabras del idioma ingl\'es como palabras clave
  en donde los lenguajes de programaci\'on tradicionales utilizan
  signos de puntuaci\'on.
\end{itemize}

\section{Notaci\'on}
A continuaci\'on presentamos una notaci\'on para especificar la
sint\'axis del lenguaje.\\
\indent Note que las definiciones que se dan a
continuaci\'on no son descripciones precisas de elementos del
lenguaje, si no herramientas que usaremos para definir la sint\'axis del
lenguaje:

\begin{itemize}
\item Una \texttt{secuencia de car\'acteres v\'alida} es aquella que
  pertenece al lenguaje regular \texttt{[a-zA-Z0-9():\_\textbackslash
    n]+}.
       
\item Un \texttt{elemento} es una \texttt{secuencia de car\'acteres
    v\'alida} encerrada entre corchetes, entre \texttt{<} y
  \texttt{>}, entre numerales o no encerrada.
       
\item Toda \texttt{secuencia de car\'acteres v\'alida} se tomar\'a
  como literal a menos que est\'e encerrada entre \texttt{<} y
  \texttt{>} o entre corchetes.

\item Todo car\'acter que este precedido por un \textbackslash se
  tomar\'a como literal, a\'un cuando no pertenezca al lenguaje de las
  \texttt{secuencias de car\'acteres v\'alidas}.
       
\item Uno o m\'as \texttt{elementos} pueden estar encerrados entre
  corchetes, esto querr\'a decir que ese grupo de \texttt{elementos},
  en ese order, es opcional en el contexto en el que se encuentra
  especificado.
       
% CHECK THIS!!
\item Si uno o m\'as \texttt{elementos} est\'an encerrados entre
  numerales, quiere decir que esos \texttt{elementos} deben ser
  considerados un solo \texttt{elemento}.
       
\item Si dos \texttt{elementos} est\'an separados por \texttt{|}
  indica que uno y solo uno de esos \texttt{elementos} puede aparecer
  en ese contexto. El \texttt{|} act\'ua como un \emph{XOR} l\'ogico.
       
\item \texttt{<tipo>} representa un tipo cualquiera del lenguaje,
  explicados en la secci\'on \ref{sec:tipos}.

\item \texttt{<Int>},\texttt{<Float>},\texttt{<Bool>},\texttt{<Char>}
  representan los tipos b\'asicos del lenguaje, descritos en
  \ref{sec:tiposbasicos}.
       
\item \texttt{<identificador>}, \texttt{<id1>}, \texttt{<id2>}...
  \texttt{<idN>} representan identificadores del lenguaje. El lenguaje
  de los identificadores est\'a explicado en la secci\'on
  \ref{sec:ident}
       
\item \texttt{<expresion>}, \texttt{<expr1>}, \texttt{<expr2>}...
  \texttt{<exprN>} representan expresiones cualquiera del lenguaje, como
  son explicadas en la secci\'on \ref{sec:expr}.
       
\item \texttt{<code\_block>}, \texttt{<code\_block1>},
  \texttt{<code\_block2>}...  \texttt{<code\_blockN>} representan
  bloques de c\'odigo del lenguaje, estos bloques de c\'odigo se
  contruyen secuenciado intrucciones y estructuras de control del
  lenguaje. La secuenciaci\'on esta explicada en la secci\'on
  \ref{sec:instr}.
       
\item \texttt{<lista\_de\_argumentos>}, \texttt{<arg1>},
  \texttt{<arg2>}...  \texttt{<argN>} representan par\'ametros formales
  para las definiciones de funciones del lenguaje. El lenguaje de los
  par\'ametros formales est\'a explicado en la secci\'on
  \ref{sec:ident}, las funciones est\'an descritas con detalle en la
  secci\'on \ref{sec:func}. Al igual que con los
  \texttt{identificadores} y \texttt{expresiones}, note la posibilidad
  de que la lista de argumentos lleve \'indices para hacer referencia a
  su tama\~no de forma abstracta.

\item Note que los n\'umeros representan \'indices sobre las listas
  explicadas en los cuatro puntos anteriores. Si existen dos listas que
  tiene \'indices iguales al comienzo y al final, asuma que ambas listas
  tienen la m\'isma cantidad de elementos.
       
\item La cantidad de espacios que deben separar a dos palabras clave
  es arbitraria, pero siempre mayor a cero. Por comodidad, en este
  documento se usa un solo espacio.
\end{itemize}


\chapter{Estructura L\'exica}

\section{Codificaci\'on del input del compilador}
\label{sec:codif}
La entrada del compilador viene en forma de car\'acteres
codificados en ASCII extendido.

\section{Estructura de una l\'inea}
\label{sec:line_struct}
Una l\'inea termina con un delimitador de l\'inea. Los elementos
de las l\'ineas est\'an separados por espacios. Los elementos de una
l\'inea (suponiendo que sea una l\'inea de un programa v\'alido del
lenguaje) son elementos sint\'acticos del lenguaje. Tanto los delimitadores
como los espacios est\'an descritos en las dos secciones siguientes.

\subsection{Delimitadores de l\'ineas}
\label{sec:line_terms}
Un delimitador de l\'inea puede ser un salto de l\'inea (ASCII 0xA),
retorno de carro (ASCII 0xD) y retorno de carro seguido de
salto de l\'inea. No existe forma de separar una l\'inea en varias.

\subsection{Espacios}
\label{sec:spaces}
Para los espacios se permiten los car\'acteres ASCII espacio,
tabulador y formfeed (ASCII's 0x20, 0x9 y 0xC). Un comentario
tambi\'en puede ser usado como espacio, los comentarios est\'an
documentados en la secci\'on \ref{sec:comments}.\\
\indent No importa el n\'umero de espacios que separan dos elementos,
son tratados como un solo espacio.

\section{Comentarios}
\label{sec:comments}
Existen dos tipos de comtentarios:

\subsection{De una sola l\'inea}
Los comentarios de una l\'inea puedes cubrir parte de una l\'inea
o la l\'inea completa. La parte de la l\'inea donde comienza el
comentario est\'a marcada con \texttt{//} y termina con el
primer salto de l\'inea desde el comienzo del comentario. El salto de
l\'inea no se incluye en el comentario, la cadena \texttt{//} que marca su
comienzo si.\\
\indent Un comentario, aunque no sea parseado por la gram\'atica ni
convertido en un token, actua como un espacio a nivel sem\'antico.

\subsection{De varias l\'ineas}
Un comentario de varias l\'ineas comienza con la cadena \texttt{/.} y
termina con la cadena \texttt{./}, ambos delimitadores est\'an incluidos
en el comentario. Un comentario de varias l\'ineas puede nacer en cualquier parte
de cualquier l\'inea y morir en cualquier parte de esa misma l\'inea;
o en cualquier parte de cualquier l\'inea posterior.

\section{Identificadores y par\'ametros formales}
\label{sec:ident}
Los identificadores y par\'ametros formales son aquellas frases que
pertenecen al lenguaje regular \texttt{[a-zA-Z][a-zA-Z0-9\_']*}. Es
decir, los identificadores deben comenzar por una letra may\'uscula o
min\'uscula, a partir de ese punto cualquier combinaci\'on de letras
may\'usculas y min\'usculas, n\'umeros, \texttt{\_} (piso) y
\texttt{'} (comillas simples) es v\'alida.

\section{Palabras clave}
\label{sec:keywords}
La lista de palabras clave de Zinara que coinciden con el lenguaje regular
de identificadores y parametros formales (secci\'on \ref{sec:ident}) es la que sigue:
\begin{itemize}
\item and
\item as
\item Bool
\item break
\item Char
\item continue
\item copy
\item def
\item default
\item elif
\item else
\item end
\item enddef
\item endfor
\item endif
\item endwhile
\item False
\item Float
\item for
\item if
\item in
\item Int
\item or
\item print
\item read
\item return
\item String
\item True
\item var
\item while
\item xor
\end{itemize}

\section{Operadores}
\label{sec:agrup}
Las cadenas para los operadores son:
\begin{itemize}
\item \texttt{+}
\item \texttt{-}
\item \texttt{*}
\item \texttt{/}
\item \texttt{\%}
\item \texttt{\^}
\item \texttt{>}
\item \texttt{<}
\item \texttt{>=}
\item \texttt{<=}
\item \texttt{==}
\item \texttt{<==>}
\item \texttt{and}
\item \texttt{or}
\item \texttt{\&}
\item \texttt{|}
\item \texttt{xor}
\item \texttt{!}
\end{itemize}

\chapter{Dise\~no de Tipos}
\label{sec:tipos}

\section{Tipos b\'asicos}
\label{sec:tiposbasicos}

Los tipos b\'asicos del lenguaje son:

\begin{itemize}
\item \textbf{Int}: Entero
\item \textbf{Float}: Flotante
\item \textbf{Bool}: Booleano
\item \textbf{Char}: Caracter 
\end{itemize}
La semantica y compatibilidad de los operadores sobre 
tipos b\'asicos est\'a documentada en la secci\'on \ref{subsec:comp-tipos}

\subsection{Operadores Aritm\'eticos}
\begin{itemize}
\item Suma (\texttt{+})
\item Resta    (\texttt{-})
\item Producto (\texttt{*})
\item Divisi\'on  (\texttt{/})
\item M\'odulo (\texttt{\%})
\item Potencia (\texttt{\^})
\item Menos unario (\texttt{-})
\end{itemize}

\subsection{Operadores Relacionales}
\begin{itemize}
\item Mayor         (\texttt{>})
\item Menor         (\texttt{<})
\item Mayor o igual (\texttt{>=})
\item Menor o igual (\texttt{<=})
\item Igual (\texttt{==})
\item Negaci\'on      (\texttt{!})
\end{itemize}

\subsection{Operadores L\'ogicos}
\begin{itemize}
\item And con corto-circuito (\texttt{and})
\item And sin corto-circuito (\texttt{\&})
\item Or con corto-circuito  (\texttt{or})
\item Or sin corto-circuito  (\texttt{|})
\item Xor (\texttt{xor})
\end{itemize}

\subsection{Precedencias y asociatividades de operadores}
\label{sec:basic_preced}
Todos los operadores asocian a la izquierda. Sus precedencias son:
\begin{itemize}
\item \texttt{and}, \texttt{\&}, \texttt{or}, \texttt{|}, \texttt{xor}
\item \texttt{<} , \texttt{>} , \texttt{<=} , \texttt{>=} , \texttt{==} , \texttt{<=>} , \texttt{!}
\item \texttt{-} (Unario)
\item \texttt{\^}
\item \texttt{*} ,  \texttt{/}, \texttt{\%}
\item \texttt{+}, \texttt{-} (Binario)
\item \texttt{Indexaci\'on} (Sobre tipos compuestos)
\end{itemize}
Las indexaciones se hacen sobre tipos compuestos (secci\'on \ref{sec:tiposcompuestos}).
Todas las indexaciones tienen la misma precedencia.

\section{Tipos compuestos}
\label{sec:tiposcompuestos}

Los tipos compuestos del lenguaje son:

\begin{itemize}
\item \textbf{String}: Cadena de car\'acteres.
\item \textbf{List}: Listas de tama\~no fijo de elementos. Los
  elementos que conforman la lista deben ser del m\'ismo tipo.
\item \textbf{Tuple}: Tuplas de tama\~no fijo que contienen elementos de
  cualquier tipo.
\item \textbf{Dictionary}: Estructuras que contienen pares de
  \emph{Clave/Valor}, d\'onde las claves son identificadores con los
  cuales se puede indexar a los valores guardados en la
  estructura. Los diccionarios son de tamaño fijo.
\end{itemize}

A continuaci\'on hablaremos brevemente de la sem\'antica y
comportamiento de estos tipos.

\subsection{Strings}
\label{sec:strings}
\begin{itemize}
\item El tipo \emph{String} est\'a implementado como una lista de
  \emph{Char}, dej\'andole la tarea al programador de construir
  funciones que le permitan manipular \emph{Strings}
\end{itemize}

\subsection{Listas}
\label{sec:lists}
\begin{itemize}
\item El tipo \emph{List} representa contenedores homogeneos,
  mutables, de tama\~no fijo que pueden contener cualquier tipo
  b\'asico o compuesto del lenguaje.
\item La indexaci\'on de las listas se hace a trav\'es de \'indices
  num\'ericos.
\item Los indices de una lista deben ser conocidos a tiempo de
  compilaci\'on.
\end{itemize}

\subsection{Tuplas}
\label{sec:tuples}
\begin{itemize}
\item El tipo \emph{Tuple} representa contenedores heterogeneos,
  mutables y de tama\~no fijo que pueden contener cualquier tipo
  b\'asico o compuesto del lenguaje.
\item Es necesario que cualquier tipo de indexamiento sobre una tupla
  se realice con indices que puedan ser conocidos a tiempo de
  compilaci\'on, para poder realizar el chequeo estatico de tipos
  correspondiente.
\item Deben tener, por lo menos, dos elementos.
\item La indexaci\'on de las tuplas se hace a trav\'es de \'indices
  num\'ericos, d\'onde el n\'umero del \'indice hace referencia al elemento
  guardado en dicha posici\'on.
\item Los \'indices de una tupla deben ser conocidos a tiempo de
  compilaci\'on.
\end{itemize}

\subsection{Diccionarios}
\label{sec:dics}
\begin{itemize}
\item El tipo \emph{Dictionary} representa contenedores heterogeneos,
  mutables y de tama\~no fijo que pueden contener cualquier tipo
  b\'asico o compuesto del lenguaje.
\item Para recuperar un valor de un diccionario (indexarlo) se
  utilizan \'indices alfanum\'ericos que pertenecen al lenguaje
  regular de los identificadores del lenguaje tal como se describen en
  la secci\'on \ref{sec:ident}.
\item Durante la declaraci\'on de un diccionario se definen los
  nombres de las entradas y el tipo correspondiente a cada una de
  ellas; cada nombre de entrada puede tener un valor asociado que se
  puede asignar despu\'es de declarado el diccionario, ese valor debe
  ser del tipo asociado al nombre de la entrada. Los nombres de las
  entradas deben pertenecer al lenguaje regular de los identificadores
  del lenguaje (Secci\'on \ref{sec:ident}).
\item Un diccionario no puede ser vac\'io, es decir, no puede no tener
  ninguna entrada.
\item No pueden ser consultados con \'indices no conocidos a tiempo de
  compilaci\'on.
\end{itemize}

\subsection{Operadores sobre tipos compuestos}

\subsubsection{Indexaci\'on de listas y tuplas}
\label{sec:index_list}
Las listas y tuplas se indexan con el operador \texttt{[]}.
Es un operador binario que recibe un entero y una lista o tupla.
Se usa de manera postfija con respecto a la lista o tupla y encierra
entre corchetes una expresi\'on entera (Expresiones: \ref{sec:expr}, 
Tipos b\'asicos: \ref{sec:tiposbasicos}).\\
\indent El operador retorna el valor que esta en la posici\'on
de la lista o tupla que contada desde cero que indica la
expresi\'on entera.
\indent Por ejemplo, para indexar la posici\'on cero
de la lista \texttt{a} el c\'odigo podr\'ia ser \texttt{a[0]}.

\subsubsection{Indexaci\'on de diccionarios}
\label{sec:index_dict}
Los diccionarios se indexan con el operador \texttt{. (punto)}. Es un operador
binario infijo que recibe a la izquierda una expresion del tipo
diccionario y a la derecha un identificador (Expresiones: \ref{sec:expr}, 
Tipos compuestos: \ref{sec:tiposcompuestos}, Identificadores: \ref{sec:ident}).\\
\indent Retorna el valor del campo del diccionario que tiene un nombre
igual al del identificador que recibe a la derecha.

\section{Tipos definidos por el usuario}
\label{sec:tiposusuario}
\emph{\textbf{Zinara}} permite al usuario declarar sus propios tipos,
así como también hacer alias de los ya existentes, de tal manera que
pueda abstraer el problema de la manera mas c\'omoda posible. Para
permitir esto \emph{\textbf{Zinara}} usa un alias profundo para ligar
un nombre definido por el usuario al sistema de tipos del lenguaje, y
un alias superficial para renombrar cualquier tipo ya existente.

\subsection{Alias Profundo o \texttt{newtype}}
\label{sec:tiposusuario:newtype}
Las variables o constantes de un tipo \texttt{A} definido por el
usuario son compatibles \'unicamente con las variables o constantes
del mismo tipo \texttt{A}.\\
\indent Un nuevo tipo se puede crear a partir de cualquier tipo
b\'asico o compuesto del lenguaje, así como también de cualquier tipo
ya previamente definido por el usuario. La sintaxis para declarar un
tipo nuevo es la que sigue:

\begin{verbatim}
newtype <tipo_nuevo> as <tipo>
\end{verbatim}

\texttt{\textless tipo\_nuevo\textgreater} representa el nombre del
tipo nuevo. Este nombre debe pertenecer al lenguaje regular
\texttt{[a-zA-Z][a-zA-Z\_0-9]*}. \texttt{\textless tipo\textgreater}
representa un tipo, b\'asico o compuesto, ya definido por el lenguaje.\\

Cuando se opera un tipo definido por el usuario con un tipo intrinseco
del lenguaje, el compilador revisa que la operaci\'on sea compatible
con los tipos subyacentes.\\
Dos tipos definidos por el usuario son siempre incompatibles, incluso
cuando los dos estan representado por el mismo tipo intrinseco del
lenguaje.\\

\subsection{Alias Superficial o \texttt{aliastype}}
Los alias de tipos en \emph{\textbf{Zinara}} representan simplemente
un seudómino que el usuario puede darle a cualquier tipo del sistema
de tipos (incluso los definidos por el usuario). Para definir un alias
se utiliza la siguiente sintaxis:

\begin{verbatim}
aliastype <alias_nuevo> as <tipo>
\end{verbatim}

Dónde \texttt{\textless alias\_nuevo\textgreater} representa el nombre
del nuevo alias. Este nombre debe pertencer al lenguaje regular
\texttt{[a-zA-Z][a-zA-Z\_0-9]*}. \texttt{\textless tipo\textgreater}
representa un tipo previamente definido en el programa.

Los alias son únicamente para beneficio de lectura del programador. El
compilador transforma cualquier alias en el tipo al que éste hace
referencia, y por ende los alias no estan ligados al sistema de tipos
de \emph{\textbf{Zinara}}.\\

\chapter{Compatiblidad de tipos}
\label{subsec:comp-tipos}
Se decidi\'o no usar un n\'umero elevado de coerciones para hacer el
sistema de tipos m\'as \'util al momento de prevenir el uso incorrecto
de variables y para ayudar al programador escribir una abstracci\'on del
problema m\'as precisa.\\

\indent Las cohersiones transforman al tipo donde se pierda menos informaci\'on.\\

\indent A continuaci\'on presentamos las tablas de compatibilidad de
tipos. Estas muestran el resultado de usar un operador binario o unario sobre
todas las posibles combinaciones de tipos b\'asicos del lenguaje m\'as
el tipo compuesto String.\\

\textbf{Tipos B\'asicos:}
\begin{itemize}
\item
\begin{tabular}{|c||ccccc|}
	\hline
\texttt{+}  &  Int    &        Float & Char         &      String  & Bool  \\
	\hline \hline
Int    & Int          & Float        & \emph{Error} & \emph{Error} & \emph{Error} \\
Float  & Float        & Float        & \emph{Error} & \emph{Error} & \emph{Error}\\
Char   & \emph{Error} & \emph{Error} & String       & String       & \emph{Error}\\
String & \emph{Error} & \emph{Error} & String       & String       & \emph{Error}\\
Bool   & \emph{Error} & \emph{Error} & \emph{Error} & \emph{Error} & \emph{Error}\\
	\hline
\end{tabular}\\
Notas: La suma de dos String los concatena, al igual que la suma de un String
con un Char y de un Char con un Char.

\item
\begin{tabular}{|c||ccccc|}
	\hline
\texttt{-,*,\%,/,\^}  &  Int  & Float & Char  & String & Bool  \\
	\hline \hline
Int    & Int   & Float & \emph{Error} & \emph{Error} & \emph{Error}\\
Float  & Float & Float & \emph{Error} & \emph{Error} & \emph{Error}\\
Char   & \emph{Error} & \emph{Error} & \emph{Error} & \emph{Error} & \emph{Error} \\
String & \emph{Error} & \emph{Error} & \emph{Error} & \emph{Error} & \emph{Error} \\
Bool   & \emph{Error} & \emph{Error} & \emph{Error} & \emph{Error} & \emph{Error} \\
	\hline
\end{tabular}\\

Notas: Para flotantes y enteros esta operaci\'on representa el m\'odulo de la
divisi\'on de los valores.\\

\item
\begin{tabular}{|c||ccccc|}
	\hline
\texttt{<,>,<=,>=,==}&  Int  & Float & Char & String & Bool  \\
	\hline \hline
Int      &  \emph{Bool} &  \emph{Bool} & \emph{Error} & \emph{Error} & \emph{Error}\\
Float    &  \emph{Bool} &  \emph{Bool} & \emph{Error} & \emph{Error} & \emph{Error}\\
Char     & \emph{Error} & \emph{Error} & \emph{Error} & \emph{Error} & \emph{Error} \\
String   & \emph{Error} & \emph{Error} & \emph{Error} & \emph{Error} & \emph{Error} \\
Bool     & \emph{Error} & \emph{Error} & \emph{Error} & \emph{Error} & \emph{Error} \\
	\hline
\end{tabular}

\item
\begin{tabular}{|c||ccccc|}
	    \hline
and,or,xor\texttt{|},\& &  Int&    Float     &     Char     &     String   & Bool  \\
	    \hline \hline
Int         & \emph{Error} & \emph{Error} & \emph{Error} & \emph{Error} & \emph{Error} \\
Float       & \emph{Error} & \emph{Error} & \emph{Error} & \emph{Error} & \emph{Error} \\
Char        & \emph{Error} & \emph{Error} & \emph{Error} & \emph{Error} & \emph{Error} \\
String      & \emph{Error} & \emph{Error} & \emph{Error} & \emph{Error} & \emph{Error} \\
Bool        & \emph{Error} & \emph{Error} & \emph{Error} & \emph{Error} & \emph{Bool} \\
	\hline
\end{tabular}
-
\item
\begin{tabular}{|c||cccc|}
 \hline
!       &  Int  & Float & Char  & Bool  \\
 \hline \hline
        & \emph{Error} & \emph{Error} & \emph{Error} & \emph{Bool} \\
	\hline
\end{tabular}
\end{itemize}

\chapter{Expresiones}
\label{sec:expr}

Las expresiones tienen un tipo asociado. El tipo de una expresi\'on es
inferido o determinado en funci\'on de los elementos que formen parte
de la misma. Se detallan a continuaci\'on:

\begin{itemize}
\item \textbf{Expresi\'on Binaria:} Es una expresion que est\'a
  determinada por dos expresiones cualquiera y un operador que las
  relaciona. El operador puede ser cualquiera de los operadores
  binarios detallados en la secci\'on \ref{sec:ops}. El tipo
  resultante de una expresi\'on binaria es obtenido utilizando las
  tablas de compatibilidad explicadas en la secci\'on \ref{subsec:comp-tipos}.

\item \textbf{Expresi\'on Unaria:} Es una expresi\'on que se encuentra
  formada por alg\'una otra expresi\'on cualquiera y un operador que
  act\'ua sobre ella. An\'alogamente que con las expresiones binarias,
  el operador puede ser cualquiera de los operadores unarios
  detallados en la secci\'on \ref{sec:ops}. Una operaci\'on bien
  definida con el operador de ``negaci\'on'' tiene como tipo asociado
  \emph{Bool} mientras que el tipo de una expresi\'on con el operador
  ``menos unitario'' es \emph{Int} si la expresi\'on asociada tiene tipo
  \emph{Int} \'o \emph{Float} si la expresi\'on asociada tiene tipo
  \emph{Float}.

\item Cualquier literal de los tipos b\'asicos del lenguaje es una
  expresi\'on. Por ejemplo \texttt{42} en el caso de los \emph{Int},
  \texttt{3.5} en el caso de los \emph{Float}, \texttt{'a'} en el caso
  de los \emph{Char}, \texttt{True} en el caso de los \emph{Bool},
  \texttt{``cadena de caracteres''} en el caso de los \emph{String},
  \texttt{[1, 2, 3]} en el caso de los \emph{List}, \texttt{(True, [1,
    2], 'a')} en el caso de los \emph{Tuple} y \texttt{{carro:
      ``mazda'', id: 6}} en el caso de los \emph{Dictionary}.

\item Cualquier identificador ya declarado de un programa puede ser
  usado como una expresi\'on que tiene el mismo tipo de la
  declaraci\'on del identificador y su valor es el valor que almacene
  dicho identificador. El comportamiento del lenguaje si se usa un
  identificador al cual no se le ha asignado ning\'un valor esta, por
  los momentos, indefinido.

\item Cualquier llamada a una funci\'on ya declarada es una
  expresi\'on que tiene como valor resultante el valor devuelto por la
  funci\'on, y su tipo es el tipo de retorno de la funci\'on.
\end{itemize}

\chapter{Declaraciones}
\label{sec:decl}

Las declaraciones se pueden realizar de dos formas:\\

\textbf{Una a la vez}:
\begin{verbatim}
[var] <tipo> <identificador> [= <expresion>]
\end{verbatim}

\textbf{o de forma m\'ultiple}:
\begin{verbatim}
[var] <tipo> <id1>, <id2>, ..., <idN> [= <expr1>, <expr2>, ..., <exprN>]
\end{verbatim}

La palabra clave \texttt{var} denota si el identificador, o los
indentificadores, que estan siendo declarados se comportar\'an como
variables en el lenguaje de programaci\'on. Si \texttt{var} se omite,
el identificador (o identificadores) ser\'a declarado como constante,
y una vez que le sea asignado un valor de una expresi\'on, \'este
ser\'a inmutable. Las palabras reservadas que pueden ir en el lugar de
\textless tipo\textgreater son, para tipos b\'asicos y los String:

\begin{itemize}
\item Int
\item Float
\item Bool
\item Char
\item String
\end{itemize}

\section{Declaraci\'on de listas, tuplas y diccionarios}
\label{sec:declc}
A continuaci\'on describimos la manera en que se especifican los tipos
de los tipos compuestos durante su declaraci\'on. La sintaxis de la
declaraci\'on es exactamente igual a la anterior, lo que se especifica
aqu\'i es la forma en que se indica el tipo compuesto a usar.\\

\textbf{List:}\\
Se declara encerrando entre corchetes el tipo de la
lista. De esta forma una lista de Int seria \texttt{[Int]}, una lista
de listas de Strings ser\'ia \texttt{[[String]]}.\\

La sintaxis es:
\begin{verbatim}
[<tipo>]
\end{verbatim}

\textbf{Tuple:}\\
Para declarar una tupla se cierra entre par\'entesis una lista de
tipos separados por comas, uno por cada elemento que la tupla vaya a
contener; el orden es importante. Una tupla de tres elementos donde el
primero es una lista de Int, el segundo un Bool y el tercero una tupla
de dos Int se escribir\'ia de la siguiente forma:
\texttt{([Int],Bool,(Int,Int))}.\\

La sintaxis es:
\begin{verbatim}
(<tipo1>,<tipo2>,...,<tipoN>)
\end{verbatim}

\textbf{Diccionarios:}\\
Para los diccionarios se deben declarar los nombres de las entradas y
los tipos asociados. Una entrada de diccionario se declara escribiendo
el nombre de la entrada, luego dos puntos, y luego el tipo
asociado. Un diccionario se declara encerrando entre llaves una lista
de declaraciones de entradas de diccionario separadas por
comas. As\'i, un diccionario con una entrada \texttt{carro} del tipo
String y otra entrada \texttt{color} del tipo Int se declarar\'ia
\texttt{\{carro:String, color:Int\}}\\

La sintaxis es:
\begin{verbatim}
{<id1>:<tipo1> , <id2>:<tipo2> ,..., <idN>:<tipoN>}
\end{verbatim}
 % PORQUE?!

Existe un tercer tipo de declaraci\'on: \emph{Declaraci\'on de
  funciones}, el cual es explicado en la secci\'on \ref{sec:func}.

\section{Alcanze}
\label{sec:scope}
Una variable s\'olo es alcanzable dentro del bloque de c\'odigo donde
es declarada. Tambi\'en es alcanzable desde los bloques de c\'odigo
internos al bloque de c\'odigo donde esta declarada siempre y cuando
no halla declaraciones de variables con un identificador igual al de
la variable.

\section{Dominio de nombres}
\label{sec:namespaces}
Las varibles, constantes, funciones y procedimientos comparten un mismo
espacio de nombres.

\chapter{Instrucciones}
\label{sec:instr}

Las instrucciones del lenguaje que se pueden realizar sobre los tipos
definidos en la secci\'on \ref{sec:tipos} y dentro de las estructuras
de control definidas en la secci\'on pasada son las siguientes:

\begin{itemize}
\item Secuenciaci\'on
\item Asignaci\'on
\item Return
\item Break
\item Continue
\item Estructuras de control (explicadas con detalle en la secci\'on \ref{sec:ecntrl})
\item Impresi\'on
\end{itemize}

Adicionalmente, las declaraciones de identificadores o funciones
tambi\'en cuentan como instrucciones del lenguaje y pueden ir en
cualquier lugar donde pueda ir una instrucci\'on.

\section{Secuenciaci\'on}
Se secuenci\'an varias instrucciones y estructuras de control
separandolas con saltos de l\'inea. No importa que la \'ultima
instrucci\'on tenga tambi\'en un salto de l\'inea.

\section{Asignaci\'on}
Sintaxis:
Las asignaciones se pueden realizar de dos formas:\\

\textbf{Una a la vez}:
\begin{verbatim}
<identificador> [= <expresion>]
\end{verbatim}

\textbf{O de forma m\'ultiple}:
\begin{verbatim}
<id1>, <id2>, ..., <idN> [= <expr1>, <expr2>, ..., <exprN>]
\end{verbatim}

\section{Return}
\label{sec:return}
Sintaxis:
\begin{verbatim}
return <expr>
\end{verbatim}

Esta instrucci\'on solo puede ser usada dentro de una funci\'on. Lo
que hace es detener el flujo de la funci\'on y hacer que esta devuelva
la expresi\'on \texttt{<expr>}. Es importante destacar que el chequeo
de tipos asegura que el tipo inferido de la expresi\'on \texttt{expr}
sea igual al tipo de retorno de la funci\'on en la que se encuentra.

\section{Break}
\label{sec:return}
Sintaxis:
\begin{verbatim}
break
\end{verbatim}

Esta instrucci\'on solo puede ser usada dentro de una
iteraci\'on. Ocasiona que la iteraci\'on deje de ejecutarse.
Si la estructura est\'a anidada dentro de otra iteraci\'on,
solo la estructura de control m\'as interna termina.

\section{Continue}
\label{sec:return}
Sintaxis:
\begin{verbatim}
continue
\end{verbatim}

Esta instrucci\'on solo puede ser usada dentro de una
iteraci\'on. Ocasiona que las instrucciones pertenecientes a la
iteraci\'on y posteriores a esta instrucci\'on sean ignoradas y se
comienza un nuevo ciclo en la estructura de control iterativa.  Al
igual que la instrucci\'on \texttt{break}, si la iteraci\'on est\'a
anidada dentro de otra iteraci\'on, solo la estructura m\'as interna
se ve afectada por esta instrucci\'on.

\section{Impresi\'on}
Sintaxis:
\begin{verbatim}
print ( <expr> )
\end{verbatim}

Imprime la expresion entre par\'entesis a la salida
est\'andar. \texttt{expr} debe ser de tipo \texttt{<Int>},
\texttt{<Float>}, \texttt{<Char>}, \texttt{<String>} ó
\texttt{<Bool>}.

\chapter{Estructuras de control}
\label{sec:ecntrl}
\emph{\textbf{Zinara}} tiene 4 estructuras de control
\begin{itemize}
\item \texttt{if}: para ramificar el flujo del programa
\item \texttt{while}: para ciclos gen\'ericos
\item \texttt{cycle}: tambi\'en para ciclos gen\'ericos
\item \texttt{for}: para iterar sobre listas del lenguaje
\end{itemize}

Los bloques de las estructuras de control son finalizados con la
palabra clave \texttt{end}. Opcionalmente, pueden ser finalizados con
un terminador espec\'ifico a ese bloque. Esto es \'util cuando existen
muchos bloques de estructuras de control anidados, ya que permite
saber exactamente que tipo de estructura se esta cerrando.\\

Estas dos opciones le permiten al programador elegir entre darle m\'as
legibilidad al c\'odigo, o simplemente tipear menos car\'acteres para
finalizar sus bloques de control de flujo.\\

Ahora se revisar\'a la sintaxis de las estructuras de control, la
notaci\'on para escribir la sint\'axis es la misma que fu\'e definida al
principio de la secci\'on \ref{sec:synsem}.

\section{If}
Sintaxis:
\begin{verbatim}
   if <expr> :
       <code_block1>
  [elif <expr> :
       <code_block2>]
   .
   .
   .
  [else :
       <code_blockN>]
   end|endif
\end{verbatim}

\section{While}
Sintaxis:
\begin{verbatim}
   while <expr> :
       <code_block>
   end|endwhile
\end{verbatim}

\section{Cycle}
Sintaxis:
\begin{verbatim}
   cycle:
     [code_block]
     [on <expr1> :
          <code_block1>]
     [on <expr2> :
          <code_block2>]
      .
      .
      .
      # default :
         <code_blockN> #|
      # on <exprn> :
         <code_blockN> #
   end|endcycle
\end{verbatim}

Esta estructura de control iterativa recorrera cada una de las
clausulas \texttt{on} probando el valor de la expresi\'on, tomar\'a la
primera que sea cierta y cuando termine la ejecuci\'on del bloque de
c\'odigo asociado a esa clausula volver\'a a recorrerlas. La
estructura se detiene cuando ninguna clausula se cumple. Opcionalmente
se puede agregar un bloque de c\'odigo al pricipio de todas las
clausulas \texttt{on} que se ejecutar\'a al pricipio de todos los
ciclos.

\section{For}
Sintaxis:
\begin{verbatim}
   for <id> in <expr> :
       <code_block>
   end
\end{verbatim}

Esta estructura esta dise\~nada para recorrer listas del lenguaje,
por lo que el valor de \texttt{expr} s\'olo puede ser una lista.
El elemento actual de la iteraci\'on solo se puede accesar a trav\'es
del identificador \texttt{id} dentro del ciclo. Ni el elemento iterado
\texttt{expr} ni el \'indice \texttt{id} pueden ser modificados dentro
de esta estructura de control.\\


\chapter{Funciones}
\label{sec:func}

Las funciones en \emph{\textbf{Zinara}} son declaradas de la siguiente
forma:

\begin{verbatim}
def <identificador>(<lista_de_argumentos>) returns \<<tipoR>\>:
[code_block]
end|enddef
\end{verbatim}

D\'onde \texttt{<tipoR>} es el tipo de retorno, y
\texttt{<lista\_de\_argumentos>} es la lista de los argumentos de la
funci\'on definida de la siguiente forma:

\begin{verbatim}
[copy] <arg1> as <tipo1>, ..., [copy] <argN> as <tipoN>
\end{verbatim}

Las funciones recibir\'an todos sus par\'ametros por referencia. En la
declaraci\'on de una funci\'on se tienen que declarar expl\'icitamente los
tipos de cada uno de los argumentos y adicionalmente se proporciona la
inclusi\'on opcional de la palabra clave \texttt{copy} antes de la
declaraci\'on de alg\'un argumento para forzar al lenguaje a hacer una
copia del valor de dicho argumento y no modificar su referencia
durante la ejecuci\'on del c\'odigo de la funci\'on.\\

Las funciones invocan de la siguiente forma:

\begin{verbatim}
<identificador>(<expr1>, ..., <exprN>)
\end{verbatim}

Las llamadas a funciones pueden ser usadas como expresiones y como
instrucciones (para el caso en que tengan efectos de borde).

\chapter{Representaci\'on de los datos}
\label{chap:impl}
\begin{itemize}
\item \textbf{Int}: Enteros de 32bits.
\item \textbf{Float}: Flotante del 64bits si la arquitectura es de
                      64bits, de 32 si la arquitectura es de 32bits.
\item \textbf{Bool}: Booleano, se representan en unos o ceros de 32bits.
\item \textbf{Char}: Caracter ASCII. Se guardan en un (1) byte.
\item \textbf{Listas}: Se guardan como arreglos lineales continuos en memoria.
\item \textbf{Tuplas}: Se guardan como arreglos lineales continuos en memoria.
\item \textbf{Diccionarios}: Se guardan como estructuras que respetan el alineamiento
                       de memoria.
\end{itemize}

\chapter{IO}
\label{chap:io}

\chapter{Requerimientos}
\label{sec:requ}

Para la compilaci\'on y ejecuci\'on del proyecto es necesario cumplir con
los siguientes requerimientos m\'inimos:

\begin{itemize}
\item M\'aquina virtual y kit de desarrollo de \emph{Java 1.4.2}.
\item Instalaci\'on est\'andar de \emph{JFlex}.
\item Instalaci\'on est\'andar de \emph{JavaCup}.
\item Instalaci\'on est\'andar de \emph{Apache Ant} (\'Unicamente para
  compilaci\'on).
\end{itemize}

\chapter{Compilaci\'on y ejecuci\'on del compilador}
\section{Compilaci\'on del compilador}
\label{sec:compilacion}
Para compilar \emph{\textbf{Zinara}}, una vez que haya sido editado
alg\'un archivo del c\'odigo fuente, se ejecutan las siguientes l\'ineas de
comando en una consola UNIX.

\begin{verbatim}
$> cd Zinara-Barreto-Jaber/project/
[$> ant clean # para limpiar la compilaci\'on anterior]
$> ant dist
\end{verbatim}

Esto se encargar\'a de compilar y empaquetar adecuadamente el proyecto,
y un ejecutable \texttt{.jar} ser\'a colocado dentro de la carpeta
\texttt{bin/}.\\

Es importante destacar que si se editan alguno de los archivos
\texttt{parser.cup} o \texttt{lexer.flex}, es necesario ejecutar sobre
ellos la herramienta \emph{JavaCUP} o la herramienta \emph{JFlex}
según sea el caso, para poder generar los archivos \texttt{.java}
correspondientes.\\

La ejecuci\'on de las herramientas \emph{JavaCUP} y \emph{JFlex} no ser\'a
documentada en este informe pu\'es existen formas diversas de hacerlo, y
var\'ian seg\'un el sistema operativo donde se ejecute y la instalaci\'on
que se haya realizado de ellas en cada computadora. Para m\'as
informaci\'on referirse a \cite{javacup} y \cite{jflex}.\\

En \cite{github} se encuentra un repositorio con el mismo c\'odigo
fuente adjunto (posiblemente m\'as actualizado) y el c\'odigo fuente de
las herramientas que estan siendo utilizadas para el desarrollo de
\emph{\textbf{Zinara}}.

\section{Ejecuci\'on del compilador}
\label{sec:zejecucion}
Los pasos para ejecutar el proyecto en una consola UNIX son los
siguientes:

\begin{verbatim}
$> cd Zinara-Barreto-Jaber/project/classes/
$> java zinara.Main archivo_de_prueba.zn
\end{verbatim}

Opcionalmente se puede ejecutar el proyecto desde el un archivo
\emph{jar} realizando los siguientes pasos:

\begin{verbatim}
$> cd Zinara-Barreto-Jaber/project/bin/
$> java -jar zinara.jar archivo_de_prueba.zn
\end{verbatim}


Es importante recordar que para la ejecuci\'on del proyecto, dada la
elecci\'on de cualquiera de los dos m\'etodos de ejecuci\'on explicados
anteriormente, es necesario tener bien establecidas las variables de
ambiente requeridas por las herramientas utilizadas. Referirse a
\cite{javacup} y \cite{jflex} para m\'as informaci\'on.

\chapter{Compilaci\'on y ejecuci\'on del c\'odigo generado}
El compilador genera un archivo \texttt{x86.asm} que se debe compilar usando
el script \texttt{znc}. Solo con ejecuarlo hace las operaciones necesarias.\\
\indent La compilaci\'on genera un archivo ejecutable x86. Se puede ejecutar
normalmente.

\chapter{Corrida de Pruebas}
\label{sec:pruebas}
Para probar la efectividad del programa tenemos varios casos de
prueba. Para correr todos los casos de prueba solo se debe ejecutar el
script \texttt{allTests.sh} (asegurandose que tenga permisos de
ejecuci\'on) que se encuentra en la carpeta \texttt{project}.

\begin{verbatim}
$> cd Zinara-Barreto-Jaber/project/
$> ./allTests.sh
\end{verbatim}

Los archivos de prueba se encuentran en la carpeta
\texttt{Zinara-Barreto-Jaber/test\_files/}.
Para que el script pueda correr adecuadamente, se deben tener las clases
de JUnit en el classpath.

\chapter{Herramientas utilizadas}
\label{sec:herramientas}
Este cap\'itulo describe brevemente las herramientas que fueron
utilizadas para la realizaci\'on del analizador sint\'actico
(\emph{parser}) y el analizador lexicogr\'afico (\emph{lexer}) de
\emph{\textbf{Zinara}}, as\'i como el resto de las herramientas que han
sido utilizadas en todo el desarrollo del lenguaje.

\begin{itemize}
\item Para generar el analizador lexicogr\'afico se utiliz\'o
  \emph{JFlex}\cite{jflex}, una herramienta que recibe una
  especificaci\'on de \emph{tokens} o palabras claves y genera c\'odigo en
  \emph{Java} que se encarga de hacer el reconocimiento de dichos
  \emph{tokens} dada una entrada de texto cualquiera.

\item Para el analizador sint\'actico se utiliz\'o la herramienta
  \emph{JavaCUP}\cite{javacup}, un generador de \emph{parsers} LALR(1)
  que recibe una especificaci\'on de gram\'atica con sintaxis similar
  al conocido \emph{YACC} y genera el aut\'omata encargado en
  reconocer dicha gram\'atica en c\'odigo \emph{Java}.\\

  Ambas herramientas est\'an bien documentadas y ofrecen facilidades
  para interactuar y comunicarse entre ellas, lo cual tuvo un gran
  peso en la decisi\'on de elegirlas para el desarrollo de
  \emph{\textbf{Zinara}}.

\item Para simplificar el proceso de compilaci\'on se utiliza
  \emph{Apache Ant}\cite{ant}, una herramienta de compilaci\'on para
  compilaci\'on dirigida en \emph{Java}.

\item Para el manejo colaborativo del c\'odigo se utiliza
  \emph{GIT}\cite{git}, un controlador de versiones distribuido
  dise\~nado para manejar grandes y peque\~nos proyectos de forma
  r\'apida.

\item Para la realizaci\'on de pruebas utilizamos framework de pruebas
  \emph{JUnit}, un framework en Java que permite correr de forma
  r\'apida, ordenada y automatizada un conjunto de pruebas unitarias
  definido.
\end{itemize}

\chapter{Detalles de implementaci\'on}
\label{sec:codigo}

Para la realizaci\'on del proyecto se organiz\'o el c\'odigo fuente en
un paquete de java llamado \texttt{zinara} el cual esta estructurado
de la siguiente manera:

\begin{itemize}
\item \texttt{zinara.ast}: Paquete d\'onde se encuentran las clases
  necesarias para representar el \'arbol sint\'actico abstracto y la
  tabla de s\'imbolos. Estas clases son explicadas con m\'as detalle a
  continuaci\'on en las secciones \ref{sec:ast} y \ref{sec:st}.

\item \texttt{zinara.lexer}: Paquete d\'onde se encuentra el archivo
  \texttt{lexer.flex}, que representa la descripci\'on lexicogr\'afica
  de los \emph{tokens} de \emph{\textbf{Zinara}}. En este paquete se
  guarda un archivo llamado \texttt{Scanner.java} que contiene el
  c\'odigo generado por la herramienta \emph{JFlex} al ser aplicada
  sobre el archivo \texttt{lexer.flex}.

\item \texttt{zinara.parser}: Paquete que contiene el archivo
  \texttt{parser.cup} que describe la gram\'atica de
  \emph{\textbf{Zinara}}. An\'alogamente al paquete
  \texttt{zinara.lexer}, en \texttt{zinara.parser} se guardan los
  archivos \texttt{parser.java} y \texttt{sym.java} que son generados
  por la aplicaci\'on de la herramienta \emph{JavaCUP} sobre el
  archivo \texttt{parser.cup}.

\item \texttt{zinara.tests}: Paquete con las clases que creamos para
  JUnit. Se encuentran separadas por categorias y cada una ataca un
  aspecto espec\'ifico del lenguaje.

\item \texttt{zinara.symtable}: Paquete con las clases que permiten
  crear, consultar y llenar las tablas de s\'imbolos del lenguaje.

\item \texttt{zinara.exceptions}: Paquete con las clases que de las
  excepciones espec\'ificas al problema que se crearon para controlar
  de forma correcta la aparici\'on de errores.

\item \texttt{zinara.semantic}: Paquete con las clases que realizan
  las partes m\'as sensibles del chequeo de tipos.


\end{itemize}

La clase principal de ejecuci\'on del proyecto se encuentra en
\texttt{zinara.Main}, la cual espera recibir como par\'ametro de entrada
un archivo de texto que contenga c\'odigo escrito para
\emph{\textbf{Zinara}}. Esta clase se encarga de invocar tanto al
analizador lexicogr\'afico como al analizador sint\'actico y ejecutarlos
adecuadamente sobre el contenido del archivo recibido. Actualmente
\'esta clase principal no realiza ning\'un tipo de chequeo extra a parte
de los mencionados.

\section{\'Arbol sint\'actico abstracto}
\label{sec:ast}

El \'arbol sint\'actico abstracto se encuentra implementado como una
jerarqu\'ia de clases que establecen la representaci\'on sint\'actica de
cada uno de los elementos del lenguaje.\\

Las clases del \'arbol sint\'actico abstracto est\'an dividas en tres
conjuntos principales: \emph{expressions} para las expresiones,
\emph{instructions} para las instrucciones y \emph{types} para los
tipos. Dentro de cada una de las dos primeras divisiones se encuentran
clases que representan sint\'acticamente las estructuras explicadas en
las secciones \ref{sec:expr} y \ref{sec:instr} respectivamente.\\

En el conjunto \emph{types} se encuentran clases que representan los 4
tipos b\'asicos mencionados en la secci\'on \ref{sec:tiposbasicos}, los 4
tipos compuestos mencionados en la secci\'on \ref{sec:tiposcompuestos} y una
clase para representar el tipo de una funci\'on que es compuesto entre
los tipos de sus argumentos y su tipo de retorno.\\

La clase \emph{Type} modela una estructura en forma de \'arbol para
poder representar los tipos compuestos. En dicho \'arbol, los tipos
b\'asicos se representan como hojas y los tipos compuestos como nodos.

\section{Tabla de s\'imbolos}
\label{sec:st}

Las clases que manejan la tabla de s\'imbolos estan en los archivos
\texttt{Symtable.java} y \texttt{SymValue.java}, y se encuentran en el paquete
\texttt{zinara.symtable}

\begin{itemize}
\item \texttt{Symtable.java}: es un encapsulador(\emph{wrapper}) para
  HashTable. Simplemente da nombres semanticamente mas coherentes con
  el manejo de s\'imbolos que los de HashTable. Los pares clave/valor
  guardados en la tabla de hash son del tipo String/SymValue.

\item \texttt{SymValue.java}: es un contenedor que guarda un Type y un
  boolean. El Type simplemente indica el tipo del objeto y el boolean
  indica si el s\'imbolo es variable o no.
\end{itemize}

\section{C\'odigo adjunto del proyecto}
\label{sec:correr}

Este proyecto tiene adjunto un archivo comprimido llamado\\

\texttt{zinara-barreto-jaber-entrega1.tar.gz}\\

y dentro de \'este archivo se encuentra una \'unica carpeta llamada
\textbf{Zinara-Barreto-Jaber/}. Una vez descomprimido el archivo se
encontrar\'a la siguiente estructura de directorios:

\begin{itemize}
 \item \textbf{labs/}: Directorio donde se guardan los laboratorios
       asignados en clases.
 \item \textbf{project/}: Directorio principal del proyecto, donde se
       encuentra el codigo fuente y los binarios para la ejecuci\'on.
 \begin{itemize}
  \item \textbf{project/bin/}: Directorio donde se guardan los
	archivos ejecutables para correr el proyecto.
  \item \textbf{project/classes/}: Directorio donde en encuentran los
	archivos \texttt{.class} del proyecto.
  \item \textbf{project/lib/}: Directorio donde se guardan las
	librerias externas usadas en el desarrollo del proyecto. Para esta
	entrega no se ha agregado ninguna.
  \item \textbf{project/src/}: Directorio donde se encuentra el c\'odigo
	fuente del proyecto.
  \item \textbf{project/test\_files/}: Directorio donde se encuentran los
	casos de prueba con los que se comprueba la efectividad de la implementaci\'on.
 \end{itemize}
 \item \textbf{reports/}: Directorio donde se guarda el c\'odigo fuente
       del presente informe, as\'i como su versi\'on en \emph{pdf}.
\end{itemize}

\chapter{Estado actual del lenguaje}
\label{chap:estado}

Actualmente \emph{\textbf{Zinara}} cuenta con un funcionamiento
correcto en su analizador lexicogr\'afico y su analizador sint\'actico. A
parte de esto tambi\'en construye el \'arbol sint\'actico abstracto y la
tabla de s\'imbolos de un programa dado. El chequeo de tipos est\'a completo.\\
\indent Respecto a la generaci\'on de c\'odigo, se cuenta con declaraciones,
asignaciones, if, while, expresiones booleanas, aritmeticas, reales y chars,
adem\'as de un IO muy limitado.\\
\indent En estos momentos existes varias funcionalidades no
implementadas en el lenguaje, estas son:
\begin{itemize}
\item Chequear que no existas returns dentro del cuerpo del main, ya que
      no tiene sentido hacer return desde el main.
\item Implementar, tanto en la gram\'atica como en el \'arbol, el manejo
      de la instrucci\'on newtype para crear tipos de datos definidos
      por el usuario.
\item La construcci\'on de las listas, tuplas y diccionarios no esta hecha,
      pero si el calculo de sus lvalues
\item IO solo puede imprimir n\'umeros de un solo d\'igito que vivan en el
      \'area est\'atica. No se puede leer.
\item Los for no funcionan correctamente debido a la construcci\'on defectuosa
      de las listas.
\end{itemize}

\bibliographystyle{plain}
\bibliography{bibliography}

\end{document}